
%%%%%%%%%%%%%%%%%%%%%%%%%%%%%%%%%%%%%%%%%%%%%%%%%%%%%%%%%%%%%%%%%%%%%%%%%
% documentclass option: pick one:
% "presentation" for powerpoint-like talk,
% "handout" for  printing,
% "trans" for printing onto transparencies
%%%%%%%%%%%%%%%%%%%%%%%%%%%%%%%%%%%%%%%%%%%%%%%%%%%%%%%%%%%%%%%%%%%%%%%%%

%\documentclass[leqno,presentation]{beamer}
%\documentclass[leqno, handout]{beamer}
%\documentclass[leqno,trans]{beamer}
\documentclass[leqno,presentation,unknownkeysallowed]{beamer}

% load standard packages
\usepackage[utf8]{inputenc}
\usepackage{mathtools}
\usepackage{amsfonts}
\usepackage{amsmath,amssymb,latexsym}
\usepackage[english]{babel}
\usepackage{tikz}
\usepackage{hyperref}
\usepackage{fancyvrb}

% ######## ##     ## ######## ##     ## ########  ######
%    ##    ##     ## ##       ###   ### ##       ##    ##
%    ##    ##     ## ##       #### #### ##       ##
%    ##    ######### ######   ## ### ## ######    ######
%    ##    ##     ## ##       ##     ## ##             ##
%    ##    ##     ## ##       ##     ## ##       ##    ##
%    ##    ##     ## ######## ##     ## ########  ######

%%%%%%%%%%%%%%%%%%%%%%%%%%%%%%%
% major themes: pick one
%%%%%%%%%%%%%%%%%%%%%%%%%%%%%%%


%\usetheme{Berkeley}
%\usetheme{PaloAlto}
%\usetheme{Copenhagen}
%\usetheme{Warsaw}
\usetheme{Darmstadt}
%\usetheme{Singapore}

% Berkeley is a standard choice for presentation,
% PaloAlto is similar to Berkeley, but with round bullets,
% Copenhagen and Warsaw are good alternative, with navigation bars at the
% top and author, title at the bottom of each slide
% Darmstadt is similar, but doesn't have author/title at bottom
% Singapore is the least "flashy" of the themes, suitable for printing


%%%%%%%%%%%%%%%%%%%%%%%%%%%%%%%
% color theme: optional,
% changes default colors
%%%%%%%%%%%%%%%%%%%%%%%%%%%%%%%

\usecolortheme[RGB={0,60,125}]{structure}
%\usecolortheme{albatross}

%%%%%%%%%%%%%%%%%%%%%%%%%%%%%%%
% font theme: optional
% changes default fonts
%%%%%%%%%%%%%%%%%%%%%%%%%%%%%%%

% no need to change anything here; for a different look,
% uncomment one of the \usefonttheme commands

%% structurebold is a good alternative for a different look
\usefonttheme{structurebold}
%\usefonttheme{professionalfonts}

%%%%%%%%%%%%%%%%%%%%%%%%%%%%%%%%%%%%%%%%%%%%%%%%%%%%%%%%%%%%%%%%%%%%%%%%%
%%% bibliography settings %%%%
\setbeamertemplate{bibliography item}[text]

%%%%%%%%%%%%%%%%%%%%%%%%%%%%%%%%%%%%%%%%%%%%%%%%%%%%%%%%%%%%%%%%%%%%%%%%%
%%%% macros %%%%

% theorem declarations
\newtheorem{type}{Type}

% hides section from TOC but not from section list at top of slides
\newcommand{\hiddensection}[1]{\stepcounter{section}\section*{{#1}}}

%%%%%%%%%%%%%%%%%%%%%%%%%%%%%%%%%%%%%%%%%%%%%%%%%%%%%%%%%%%%%%%%%%%%%%%%%

\title{Spiking neural networks to model Hydra nerve nets}
\author[M. Ivanitsky, C. Puritz]{Michael Ivanitsky, Connor Puritz}
\date{\scriptsize Math 463\\ December 5, 2018}
\institute{\includegraphics[width=0.35\textwidth]{LSA_logo_1000.png}
\hspace{2em}}

%%%%%%%%%%%%%%%%%%%%%%%%%%%%%%%%%%%%%%%%%%%%%%%%%%%%%%%%%%%%%%%%%%%%%%%%%
%% logo, shows up  in top left corner of each frame
\logo{\includegraphics[width=0.4in]{umich.png} }

%%%%%%%%%%%%%%%%%%%%%%%%%%%%%%%%%%%%%%%%%%%%%%%%%%%%%%%%%%%%%%%%%%%%%%%%%

\begin{document}
%%%%%%%%%%%%%%%%%%%%%%%%%%%%%%%%%%%%%%%%%%%%%%%%%%%%%%%%%%%%%%%%%%%%%%%%%
% ######## #### ######## ##       ########
%    ##     ##     ##    ##       ##
%    ##     ##     ##    ##       ##
%    ##     ##     ##    ##       ######
%    ##     ##     ##    ##       ##
%    ##     ##     ##    ##       ##
%    ##    ####    ##    ######## ########
%%%%%%%%%%%%%%%%%%%%%%%%%%%%%%%%%%%%%%%%%%%%%%%%%%%%%%%%%%%%%%%%%%%%%%%%%

\begin{frame}
\titlepage
\end{frame}
%%%%%%%%%%%%%%%%%%%%%%%%%%%%%%%%%%%%%%%%%%%%%%%%%%%%%%%%%%%%%%%%%%%%%%%%%

%%%%%%%%%%%%%%%%%%%%%%%%%%%%%%%%%%%%%%%%%%%%%%%%%%%%%%%%%%%%%%%%%%%%%%%%%
% table of contents
%%%%%%%%%%%%%%%%%%%%%%%%%%%%%%%%%%%%%%%%%%%%%%%%%%%%%%%%%%%%%%%%%%%%%%%%%

\begin{frame}{Table of Contents}
\tableofcontents[section]
\end{frame}
%%%%%%%%%%%%%%%%%%%%%%%%%%%%%%%%%%%%%%%%%%%%%%%%%%%%%%%%%%%%%%%%%%%%%%%%%

%%%%%%%%%%%%%%%%%%%%%%%%%%%%%%%%%%%%%%%%%%%%%%%%%%%%%%%%%%%%%%%%%%%%%%%%%
%%%%%%%%%%%%%%%%%%%%%%%%%%%%%%%%%%%%%%%%%%%%%%%%%%%%%%%%%%%%%%%%%%%%%%%%%
% Section 1
\section{Artifical Neural Networks}
%%%%%%%%%%%%%%%%%%%%%%%%%%%%%%%%%%%%%%%%%%%%%%%%%%%%%%%%%%%%%%%%%%%%%%%%%
%%%%%%%%%%%%%%%%%%%%%%%%%%%%%%%%%%%%%%%%%%%%%%%%%%%%%%%%%%%%%%%%%%%%%%%%%

\begin{frame}{Slide Title}
% Useful format
\begin{definition}
Introduce ANNs, why they suck for bio, introduce SNNs, mention why they're better.
\end{definition}
\end{frame}

%%%%%%%%%%%%%%%%%%%%%%%%%%%%%%%%%%%%%%%%%%%%%%%%%%%%%%%%%%%%%%%%%%%%%%%%%
%%%%%%%%%%%%%%%%%%%%%%%%%%%%%%%%%%%%%%%%%%%%%%%%%%%%%%%%%%%%%%%%%%%%%%%%%
% Section 2
\section{Biology of Hydra vulgaris}
%%%%%%%%%%%%%%%%%%%%%%%%%%%%%%%%%%%%%%%%%%%%%%%%%%%%%%%%%%%%%%%%%%%%%%%%%
%%%%%%%%%%%%%%%%%%%%%%%%%%%%%%%%%%%%%%%%%%%%%%%%%%%%%%%%%%%%%%%%%%%%%%%%%

\begin{frame}{Evolutionary History of H. vulgaris}
\begin{itemize}
\item \textit{Hydra} are small, freshwater hydrozoans (family Cnidaria)
\item Believed to have originated around 60 Mya \cite{age}
\item Cnidarians first appeared around 580 Mya, haven't changed much since
\item Ideal for studying development of common features across animals
\end{itemize}
\begin{figure}
\center
\includegraphics[scale=0.20]{hydra.png}
\end{figure}
\end{frame}

\begin{frame}{Nerve Nets}
\begin{itemize}
\item \textit{Hydra} have diffuse nerve nets rather than a CNS
\item Once mature, a constant density gradient of neurons is maintained \cite{density}
\item 
\end{itemize}
\end{frame}

%%%%%%%%%%%%%%%%%%%%%%%%%%%%%%%%%%%%%%%%%%%%%%%%%%%%%%%%%%%%%%%%%%%%%%%%%
%%%%%%%%%%%%%%%%%%%%%%%%%%%%%%%%%%%%%%%%%%%%%%%%%%%%%%%%%%%%%%%%%%%%%%%%%
% Section 3
\section{Theoretical Framework}
%%%%%%%%%%%%%%%%%%%%%%%%%%%%%%%%%%%%%%%%%%%%%%%%%%%%%%%%%%%%%%%%%%%%%%%%%
%%%%%%%%%%%%%%%%%%%%%%%%%%%%%%%%%%%%%%%%%%%%%%%%%%%%%%%%%%%%%%%%%%%%%%%%%

\begin{frame}{Title 3}
How we'll adapt the model for Hydra
\end{frame}

\begin{frame}[fragile]{Neuron Model}
\begin{itemize}
\item Leaky integrate-and-fire model:
\begin{equation*}
\frac{dV_{m}}{dt}=\frac{1}{C_{m}}\left(-\frac{(V_{m}-V_{m}^{eq})}{R_{m}}+I_{ext}\right)
\end{equation*}
\item Computationally simpler than Hodgkin-Huxley
\item Models neuron as RC circuit with leak term
\item Doesn't explicitly specify spiking behavior or refractory period, but easy to implement using iterative ODE methods
\item Possible implementation:
\begin{figure}
\begin{BVerbatim}[fontsize=\scriptsize]
if V(t+1) > threshold:
   V(t)   <- spike
   V(t+1) <- hyperpolarize
if t < refractory period:
   V(t+1) <- hyperpolarize
\end{BVerbatim}
\end{figure}
\end{itemize}
\end{frame}

\begin{frame}{Antagonistic Neural Circuits}
\begin{itemize}
\item Assume each neuron of RP1 emits an inhibitory neurotransmitter $E_{RP1}$ when spiking, and similarly for CB with $E_{CB}$. Using normalized concentrations, the model is:
\begin{align*}
\frac{dV_{RP1}}{dt}&=\frac{1}{C_{RP1}}\left(-\frac{(V_{RP1}-V_{RP1}^{eq})}{R_{RP1}}+I_{ext}\left(1-E_{CB}\right)\right)\\
\frac{dV_{CB}}{dt}&=\frac{1}{C_{CB}}\left(-\frac{(V_{CB}-V_{CB}^{eq})}{R_{CB}}+I_{ext}\left(1-E_{RP1}\right)\right)\\
\frac{dE_{RP1}}{dt}&=d_{RP1}E_{RP1}\hspace{2em}\frac{dE_{CB}}{dt}=d_{CB}E_{CB}
\end{align*}
\end{itemize}
\end{frame}

\begin{frame}{Antagonistic Neural Circuits (cont.)}

\end{frame}


%%%%%%%%%%%%%%%%%%%%%%%%%%%%%%%%%%%%%%%%%%%%%%%%%%%%%%%%%%%%%%%%%%%%%%%%%
%%%%%%%%%%%%%%%%%%%%%%%%%%%%%%%%%%%%%%%%%%%%%%%%%%%%%%%%%%%%%%%%%%%%%%%%%
% Section 4
\section{Optimizations}
%%%%%%%%%%%%%%%%%%%%%%%%%%%%%%%%%%%%%%%%%%%%%%%%%%%%%%%%%%%%%%%%%%%%%%%%%
%%%%%%%%%%%%%%%%%%%%%%%%%%%%%%%%%%%%%%%%%%%%%%%%%%%%%%%%%%%%%%%%%%%%%%%%%

\begin{frame}{Title 4}
Maybe tensor stuff?
\end{frame}

%%%%%%%%%%%%%%%%%%%%%%%%%%%%%%%%%%%%%%%%%%%%%%%%%%%%%%%%%%%%%%%%%%%%%%%%%
%%%%%%%%%%%%%%%%%%%%%%%%%%%%%%%%%%%%%%%%%%%%%%%%%%%%%%%%%%%%%%%%%%%%%%%%%
% Section 5
\section{Future Work}
%%%%%%%%%%%%%%%%%%%%%%%%%%%%%%%%%%%%%%%%%%%%%%%%%%%%%%%%%%%%%%%%%%%%%%%%%
%%%%%%%%%%%%%%%%%%%%%%%%%%%%%%%%%%%%%%%%%%%%%%%%%%%%%%%%%%%%%%%%%%%%%%%%%

\begin{frame}{Title 5}
It'd be cool to build this, gather real data, and test. Cool to adapt to larger and more complex organisms.
\end{frame}

%%%%%%%%%%%%%%%%%%%%%%%%%%%%%%%%%%%%%%%%%%%%%%%%%%%%%%%%%%%%%%%%%%%%%%%%%
%%%%%%%%%%%%%%%%%%%%%%%%%%%%%%%%%%%%%%%%%%%%%%%%%%%%%%%%%%%%%%%%%%%%%%%%%
% Section 6
\hiddensection{References}
%%%%%%%%%%%%%%%%%%%%%%%%%%%%%%%%%%%%%%%%%%%%%%%%%%%%%%%%%%%%%%%%%%%%%%%%%
%%%%%%%%%%%%%%%%%%%%%%%%%%%%%%%%%%%%%%%%%%%%%%%%%%%%%%%%%%%%%%%%%%%%%%%%%

\begin{frame}{References}
\begin{thebibliography}{10}
\bibitem{imaging} Hydra imaging
\bibitem{density} Hydra constant density
\bibitem{age} Hydra origination date
\end{thebibliography}
\end{frame}

%%%%%%%%%%%%%%%%%%%%%%%%%%%%%%%%%%%%%%%%%%%%%%%%
\end{document}